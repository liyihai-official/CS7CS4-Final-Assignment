\chapter{Appendix}
\section{Code}
\subsection{loaddata.py}\label{Metropolis_Uniform}
\lstset{style=PythonStyle}
\begin{lstlisting}
import numpy as np
import pandas as pd
import os
from tqdm import tqdm
from datetime import datetime

# This the file loading all data and return them in 3 type classes (before during after) of pademic period
def generate_filenames(start_year, end_year, pattern):
    filenames = []
    if pattern == 0:
        # Monthly data filenames
        for year in range(start_year, end_year + 1):
            for month in range(1, 13):
                filename = f"dublinbike-historical-data-{year}-{month:02d}.csv"
                filenames.append(filename)
    elif pattern == 1:
        # Quarterly data filenames
        quarters = [(1, 4), (4, 7), (7, 10), (10, 1)]
        for year in range(start_year, end_year):
            for start_month, end_month in quarters:
                start_date = f"{year}{start_month:02d}01"
                end_year_shift = year if end_month != 1 else year + 1
                end_date = f"{end_year_shift}{end_month:02d}01"
                filename = f"dublinbikes_{start_date}_{end_date}.csv"
                filenames.append(filename)

    return filenames


def read_data():
    start_year = 2018
    end_year = 2023
    df = pd.DataFrame()
    # Choose 0 for monthly data, 1 for quarterly data
    for pattern in range(2):
        file_list = generate_filenames(start_year, end_year, pattern)
        for file in file_list: # Loading data
            file_name = "datafiles/" + file
            if os.path.exists(file_name): 
                print(f"Load file: {file_name}")
                data = pd.read_csv(file_name)

                # Unify the colunm indexes
                data.rename(columns = {"AVAILABLE BIKE STANDS": "AVAILABLE_BIKE_STANDS"}, inplace=True)
                data.rename(columns = {"AVAILABLE BIKES": "AVAILABLE_BIKES"}, inplace=True)
                data.rename(columns = {"BIKE STANDS": "BIKE_STANDS"}, inplace=True)
                
                df = pd.concat([df, data], axis=0, ignore_index=True)
    
    station_info_dict = {}
    for id in df["STATION ID"]:
        if id in station_info_dict:
            continue
        else:
            station_info_dict[id] = {
                "NAME": df[df["STATION ID"] == id]["NAME"].values[0],
                "ADDRESS": df[df["STATION ID"] == id]["ADDRESS"].values[0],
                "LATITUDE": df[df["STATION ID"] == id]["LATITUDE"].values[0],
                "LONGITUDE": df[df["STATION ID"] == id]["LONGITUDE"].values[0]
            }
    station_info_dict = dict(sorted(station_info_dict.items(), key = lambda item: item[0]))
    
    df["TIME"] = pd.to_datetime(df["TIME"])
    # Split March 2020 into pre and post pandemic
    # March 13th, the first day schools were closed 
    # (https://www.irishtimes.com/health/2023/05/05/covid-emergency-is-over-20-key-moments-of-pandemic-that-changed-the-world/)
    timepoint_begin = pd.Timestamp('2020-03-13 00:00:00')
    
    # Split January 2022 into pre and post pandemic
    # Jan 28th 2022, the day the HSE stopped releasing COVID-19 figures 
    # (https://www.irishtimes.com/health/2023/05/05/covid-emergency-is-over-20-key-moments-of-pandemic-that-changed-the-world/)
    timepoint_end = pd.Timestamp('2022-01-27 23:59:59')

    df = df.drop_duplicates()
    df_pre = df[df["TIME"] < timepoint_begin]
    df_dur = df[(df["TIME"] >= timepoint_begin) & (df["TIME"] <= timepoint_end)]
    df_post = df[df["TIME"] > timepoint_end]
    
    return df_pre, df_dur, df_post, station_info_dict

# Round a time string to the nearest 5 minutes in a datetime object
def round_to_5_minutes(time_stamp: pd.Timestamp) -> datetime:
    # Parse the input string into a datetime object
    dt = time_stamp.to_pydatetime()

    # Round down to the nearest 5 minutes
    rounded_dt = datetime(dt.year, dt.month, dt.day, dt.hour, (dt.minute // 5) * 5)

    return rounded_dt

# Round a time string to the nearest 8 hour in a datetime object
def round_to_hour(time_stamp: pd.Timestamp) -> datetime:
    # Parse the input string into a datetime object
    dt = time_stamp.to_pydatetime()

    # Round down to the nearest day
    rounded_dt = datetime(dt.year, dt.month, dt.day, (dt.hour // 8) * 8)

    return rounded_dt
    
# Round a time string to the nearest a day in a datetime object
def round_to_day(time_stamp: pd.Timestamp) -> datetime:
    # Parse the input string into a datetime object
    dt = time_stamp.to_pydatetime()

    # Round down to the nearest day
    rounded_dt = datetime(dt.year, dt.month, dt.day)

    return rounded_dt
    
# Combine time/dates into one value per time
def combine_dates(df: pd.DataFrame, period: str) -> pd.DataFrame:
    # Round times to nearest 5 minutes
    # tqdm.pandas(desc=f"Rounding {period} times to closest 5 minutes.")
    # df['TIME'] = df['TIME'].progress_apply(round_to_5_minutes)

    # Round times to nearest a day
    # tqdm.pandas(desc=f"Rounding {period} times to closest day.")
    # df['TIME'] = df['TIME'].progress_apply(round_to_day)

    # # Round times to nearest an hour
    tqdm.pandas(desc=f"Rounding {period} times to closest hour.")
    df['TIME'] = df['TIME'].progress_apply(round_to_hour)

    # Aggregate times together, with all counts summed
    return df.groupby(df['TIME'], as_index=False).aggregate({'BIKE_STANDS': 'mean', 'AVAILABLE_BIKE_STANDS': 'mean', 'AVAILABLE_BIKES': 'mean'})

# Combine time/dates of each station into one value per time
def combine_dates_station(df: pd.DataFrame, info: dict, period: str) -> pd.DataFrame:
    df_new = pd.DataFrame()
    for station in info:
        temp = combine_dates(df[df["STATION ID"] == station], period)
        temp["STATION_ID"] = station
        df_new = pd.concat([df_new, temp], axis=0, ignore_index=True)
    return df_new


# Clean data, fill zeros to un-occured station
from itertools import product
def clean_data(df:pd.DataFrame, info: dict) -> pd.DataFrame:
    station_ids = list(info.keys()) # 从字典中提取站点ID

    # 步骤 1: 创建完整的时间序列
    min_time = df['TIME'].min()
    max_time = df['TIME'].max()
    all_times = pd.date_range(start=min_time, end=max_time, freq='8H') #D for day, H for hour, T for minute

    # 为每个站点创建完整的时间序列
    full_df = pd.DataFrame(list(product(station_ids, all_times)), columns=['STATION_ID', 'TIME'])

    # 合并数据集
    df = df.set_index(['STATION_ID', 'TIME'])
    full_df = full_df.set_index(['STATION_ID', 'TIME'])
    combined_df = full_df.join(df, how='left')

    # 填充缺失值
    combined_df.fillna(method='ffill', inplace=True)
    combined_df.reset_index(inplace=True)
    return combined_df

import json
def main():
    df_pre, df_dur, df_post, station_info = read_data()
    df_pre, df_dur, df_post = combine_dates_station(df_pre, station_info, "pre-pandemic"), combine_dates_station(df_dur, station_info, "pandemic"), combine_dates_station(df_post, station_info, "post-pandemic")
    
    df_pre, df_dur, df_post = clean_data(df_pre, station_info), clean_data(df_dur, station_info), clean_data(df_post, station_info)  

    # 存储和读取 pandas DataFrame,
    df_pre.to_hdf("pre.h5", key='df', mode='w')
    df_dur.to_hdf("dur.h5", key='df', mode='w')
    df_post.to_hdf("post.h5", key='df', mode='w')

    # 将字典写入 JSON 文件
    with open('station_info.json', 'w') as json_file:
        json.dump(station_info, json_file)

if __name__ == "__main__":
    main()
    

\end{lstlisting}
\subsection{preprocessing.py}\label{Metropolis_Uniform}
\lstset{style=PythonStyle}
\begin{lstlisting}
import numpy as np
import pandas as pd

import json
# 从 JSON 文件读取数据到字典
with open('station_info.json', 'r') as json_file:
    station_info = json.load(json_file)

# Feature engineering
def get_featured_data(df, station_info):
    def get_flow(df):
        features = ['STATION_ID', 'TIME', 'TAKE_BIKES']
        data = pd.DataFrame(index=range(df.shape[0] - 1),columns=features)
    
        for i in range(1, df.shape[0]):
            data.iloc[i-1]['STATION_ID'] = df.iloc[i-1]["STATION_ID"]
            data.iloc[i-1]['TIME'] = df.iloc[i-1]["TIME"]
        
            y1 = df.iloc[i-1]["AVAILABLE_BIKE_STANDS"] - df.iloc[i]["AVAILABLE_BIKE_STANDS"]
            y2 = df.iloc[i-1]["AVAILABLE_BIKES"] - df.iloc[i]["AVAILABLE_BIKES"]
    
            # data.iloc[i-1]['BRING_BIKE_STANDS'] = y1
            # data.iloc[i-1]['TAKE_USING'] = y2 + y1
            
            data.iloc[i-1]['TAKE_BIKES'] = y2 + y2 + y1
        return data

    
    for idx, id in enumerate(station_info):
        if idx == 0:
            y = get_flow(df[df["STATION_ID"] == int(id)])
        else:
            y = pd.concat([y, get_flow(df[df["STATION_ID"] == int(id)])], axis = 0)
    y.index = range(y.shape[0])
    return y


def get_trainable_data(df, isolate_station, period):
    # Get the start-end TimeStamps
    if period == 'pre':
        start = pd.to_datetime("01-08-2018", format='%d-%m-%Y')
        end = pd.to_datetime("13-03-2020", format='%d-%m-%Y')
    elif period == 'dur':
        start = pd.to_datetime("13-03-2020", format='%d-%m-%Y')
        end = pd.to_datetime("27-01-2022", format='%d-%m-%Y')
    elif period == 'post':
        start = pd.to_datetime("27-01-2022", format='%d-%m-%Y')
        end = pd.to_datetime("25-12-2023", format='%d-%m-%Y')
    elif period == 'cross_validation':
        start = pd.to_datetime("01-08-2018", format='%d-%m-%Y')
        end = pd.to_datetime("31-10-2020", format='%d-%m-%Y')

    
    # Get full time index
    t_full = pd.array(pd.DatetimeIndex(df.iloc[:,1]).astype(np.int64)) / 1e9
    t_start = pd.DataFrame([start]).astype(np.int64) / 1e9
    t_end = pd.DataFrame([end]).astype(np.int64) / 1e9
    
    t = np.extract([np.asarray(t_full >= t_start[0][0]) &  np.asarray(t_full <= t_end[0][0])], t_full)
    # t.shape
    
    # Get the STATION_ID 
    id = np.extract([np.asarray((t_full>=t_start[0][0])) & np.asarray((t_full<=t_end[0][0]))], df.iloc[:,0]) 

    
    # Get the sampling time period
    dt = t[id == 2][1] - t[id == 2][0]
    # print("Data sampling interval is %d secs." %dt)

    t = (t - t[0]) / 60 / 60 / 24 # convert timestamp to days
    
    y = np.extract([np.asarray((t_full>=t_start[0][0])) & np.asarray((t_full<=t_end[0][0]))], df.iloc[:,2]).astype(np.float64)
    # y.shape
    y = (y - y.mean())/y.std()
    
    if isolate_station:
        y_2d = []
        t_2d = []
        for i in station_info:
            y_2d.append(y[id == int(i)])
            t_2d.append(t[id == int(i)])
        y_2d = np.array(y_2d)
        t_2d = np.array(t_2d)
        return y_2d, t_2d, id, dt
    else:
        return y, t, id, dt


def main(period, isolate_station):
    # isolate_station = False for 1-dim y, True for 2-dim y
    if period == 'pre':
        df_pre = pd.read_hdf('pre.h5', 'df')
        df = get_featured_data(df_pre, station_info)
    elif period == 'dur':
        df_dur = pd.read_hdf('dur.h5', 'df')
        df = get_featured_data(df_dur, station_info)
    elif period == 'post':
        df_post = pd.read_hdf('post.h5', 'df')
        df = get_featured_data(df_post, station_info)
    elif period == 'cross_validation':
        df_pre = pd.read_hdf('pre.h5', 'df')
        df = get_featured_data(df_pre, station_info)
        
    y, t, id, dt = get_trainable_data(df, isolate_station, period)
    
    return y, t, id, dt, station_info
\end{lstlisting}


\subsection{train\_LSTM.py}
\begin{lstlisting}
import pandas as pd
import numpy as np
import sys, math
import matplotlib.pyplot as plt
import tensorflow as tf
epoch = 10

plt.rc('font', size=18); plt.rcParams['figure.constrained_layout.use'] = True

import preprocessing
isolate_station = False
y, t, id, dt, station_info = preprocessing.main('cross_validation', isolate_station)

#plot extracted data
# plt.scatter(t[id==10], y[id==10], c='b', marker='+',s=2)
# plt.scatter(t[id==4], y[id==4], c='r', marker='+',s=2); plt.show()


def feature_all_time_series(q = 3, lag = 3):
    def feature_time_series(q, lag, plot, y, t, id, dt):
    # q-step ahead prediction
        stride = 1

        # m = math.floor(30*7*24*60*60 / dt) # number of samples per month
        w = math.floor(7*24*60*60 / dt) # number of samples per week
        d = math.floor(24*60*60 / dt)

        len = y.size - w - lag * w - q

        XX = y[q: q+len: stride]
            
        for i in range(1, lag):
            temp = y[i*w+q: i*w+q+len: stride]
            XX = np.column_stack((XX, temp))

        for i in range(0, lag):
            temp = y[i*d+q: i*d+q+len: stride]
            XX = np.column_stack((XX, temp))

        for i in range(0, lag):
            temp = y[i+q: i+q+len: stride]
            XX = np.column_stack((XX, temp))

        yy = y[lag*w+w+q: lag*w+w+q+len: stride]
        tt = t[lag*w+w+q: lag*w+w+q+len: stride]
        iidd = id[lag*w+w+q: lag*w+w+q+len: stride]
        return XX, yy, tt, iidd

    for idx, idd in enumerate(station_info):
        idd = int(idd)
        if idd == 1:
            X, Y, T, ID = feature_time_series(q, lag, True, y[id==idd], t[id==idd], id[id==idd], dt)
        else:
            X0, y0, t0, id0 = feature_time_series(q, lag, True, y[id==idd], t[id==idd], id[id==idd], dt)
            X = np.row_stack((X, X0))
            Y = np.concatenate((Y, y0))
            T = np.concatenate((T, t0))
            ID = np.concatenate((ID, id0))
    X.shape, Y.shape, T.shape, ID.shape
    return X, Y, T, ID





from sklearn.model_selection import KFold
from sklearn.metrics import mean_squared_error
from sklearn.preprocessing import PolynomialFeatures
def ridge_model():
    def regression_model(C, input_shape, name_model):
        # 参数设置
        input_shape = [input_shape]  # 根据你的数据集输入特征的数量

        if name_model == "Lasso":
            regularizer = tf.keras.regularizers.l1(1/(2*C))
        else:
            regularizer = tf.keras.regularizers.l2(1/(2*C))

        # 创建一个简单的线性模型
        model = tf.keras.models.Sequential([
            tf.keras.layers.Dense(
                units=1,  # 只有一个单元,表示是线性模型
                input_shape=input_shape,
                activation='linear',  # 线性激活函数
                kernel_regularizer=regularizer  # L1 L2 正则化
            )
        ])
        
        # 编译模型
        model.compile(optimizer=tf.keras.optimizers.Adam(1e-3),  # 随机梯度下降优化器
                    loss='mean_squared_error',
                    metrics=['mse']
                    )  # 均方误差损失函数
        model.summary()
        return model
            
    def k_fold_cross_validation(X, y, C_vals, p, name_model):
        # Initializing the MSE and standard error
        mean_error = []
        std_error = []
        for C in C_vals:
            # Polynomial Featuring
            XX = PolynomialFeatures(p).fit_transform(X)
                    
            mean_square_error_temp = []
            kf = KFold(n_splits=5)
            # Training the model and applying teh cross validation
            for train, test in kf.split(XX):
                # Choosing a model
                # model = Lasso(alpha=1/(2*C), max_iter=10000) if name_model =="Lasso" else Ridge(alpha=1/(2*C))
                model = regression_model(C, XX.shape[1], name_model)
                
                # 训练模型
                model.fit(XX[train], y[train], 
                        epochs=epoch, 
                        batch_size=32, 
                        validation_split=0.2,
                        verbose=2
                        )  # epochs和batch_size根据需要调整
                
                predictions = model.predict(XX[test])
                mean_square_error_temp.append(mean_squared_error(y[test],predictions))
            mean_error.append(np.array(mean_square_error_temp).mean())
            std_error.append(np.array(mean_square_error_temp).std())

        return mean_error, std_error
    
    mmse_p1, mstde_p1 = k_fold_cross_validation(X, Y, C_vals = [1e-4, 1e-3, 1e-2, 1, 10], p=1, name_model="Lasso")
    mmse_p2, mstde_p2 = k_fold_cross_validation(X, Y, C_vals = [1e-4, 1e-3, 1e-2, 1, 10], p=2, name_model="Lasso")
    mmse_p1_r, mstde_p1_r = k_fold_cross_validation(X, Y, C_vals = [1e-4, 1e-3, 1e-2, 1, 10], p=1, name_model="Ridge")
    mmse_p2_r, mstde_p2_r = k_fold_cross_validation(X, Y, C_vals = [1e-4, 1e-3, 1e-2, 1, 10], p=2, name_model="Ridge")

    return mmse_p1, mstde_p1, mmse_p2, mstde_p2, mmse_p1_r, mstde_p1_r, mmse_p2_r, mstde_p2_r


import tensorflow.keras as keras
import tensorflow.keras.layers as layers
def lstm_model():
    
    def lstm_model(n_features, n_units):
        # 构建模型
        model = keras.Sequential()
        model.add(layers.LSTM(n_units, input_shape=(1, n_features)))
        model.add(layers.Dense(1))  # 因为您的目标输出是单个值

        model.compile(optimizer='adam', loss='mse', metrics=['mse']) # 选择合适的优化器和损失函数
        # model.summary()
        return model
            
    def k_fold_cross_validation(X, y, n_units):
        n_features = X.shape[2]
        
        # Initializing the MSE and standard error
        mean_error = []
        std_error = []
        for n_unit in n_units:
            mean_square_error_temp = []
            kf = KFold(n_splits=5)
            # Training the model and applying teh cross validation
            for train, test in kf.split(X):
                model = lstm_model(n_features, n_unit)
                
                # 训练模型
                model.fit(X[train], y[train], 
                        epochs=epoch, 
                        batch_size=32, 
                        validation_split=0.2,
                        verbose=2
                        )  # epochs和batch_size根据需要调整
                
                predictions = model.predict(X[test])
                mean_square_error_temp.append(mean_squared_error(y[test],predictions))
            mean_error.append(np.array(mean_square_error_temp).mean())
            std_error.append(np.array(mean_square_error_temp).std())

        return mean_error, std_error
    
    
    X_LSTM = X.reshape((X.shape[0], 1, X.shape[1]))
    mse_lstm, stde_lstm = k_fold_cross_validation(X_LSTM, Y, [1,10,50,100,1000])
    return mse_lstm, stde_lstm
    
def main():
    print("Start 5-fold cross validation to Regression Models")
    mmse_p1, mstde_p1, mmse_p2, mstde_p2, mmse_p1_r, mstde_p1_r, mmse_p2_r, mstde_p2_r = ridge_model()
    print("Start 5-fold cross validation to LSTM Models")
    mse_lstm, stde_lstm = lstm_model()
    print("SUCCESS")

if __name__ == '__main__':
    main()
\end{lstlisting}

\begin{lstlisting}
import pandas as pd
import numpy as np
import sys, math
import matplotlib.pyplot as plt

plt.rc('font', size=18); plt.rcParams['figure.constrained_layout.use'] = True

import preprocessing

isolate_station = False
y, t, id, dt, station_info = preprocessing.main('pre', isolate_station)


def feature_all_time_series(q = 3, lag = 3):
    def feature_time_series(q, lag, plot, y, t, id, dt):
    # q-step ahead prediction
        stride = 1

        # m = math.floor(30*7*24*60*60 / dt) # number of samples per month
        w = math.floor(7*24*60*60 / dt) # number of samples per week
        d = math.floor(24*60*60 / dt)

        len = y.size - w - lag * w - q

        XX = y[q: q+len: stride]
            
        for i in range(1, lag):
            temp = y[i*w+q: i*w+q+len: stride]
            XX = np.column_stack((XX, temp))

        for i in range(0, lag):
            temp = y[i*d+q: i*d+q+len: stride]
            XX = np.column_stack((XX, temp))

        for i in range(0, lag):
            temp = y[i+q: i+q+len: stride]
            XX = np.column_stack((XX, temp))

        yy = y[lag*w+w+q: lag*w+w+q+len: stride]
        tt = t[lag*w+w+q: lag*w+w+q+len: stride]
        iidd = id[lag*w+w+q: lag*w+w+q+len: stride]
        return XX, yy, tt, iidd

    for idx, idd in enumerate(station_info):
        idd = int(idd)
        if idd == 1:
            X, Y, T, ID = feature_time_series(q, lag, True, y[id==idd], t[id==idd], id[id==idd], dt)
        else:
            X0, y0, t0, id0 = feature_time_series(q, lag, True, y[id==idd], t[id==idd], id[id==idd], dt)
            X = np.row_stack((X, X0))
            Y = np.concatenate((Y, y0))
            T = np.concatenate((T, t0))
            ID = np.concatenate((ID, id0))
    X.shape, Y.shape, T.shape, ID.shape
    return X, Y, T, ID



X, Y, T, ID = feature_all_time_series(q = 3, lag = 3)

# 在训练之前,确保 x_train 和 x_test 的形状是 (样本数, 1, 特征数)
X_LSTM = X.reshape((X.shape[0], 1, 9))
X_LSTM.shape


from sklearn.model_selection import train_test_split
x_train, x_test, y_train, y_test = train_test_split(X_LSTM, Y, test_size=0.2)
x_train.shape, x_test.shape, y_train.shape, y_test.shape

import tensorflow as tf
import tensorflow.keras as keras
import tensorflow.keras.layers as layers

from tensorflow.keras.callbacks import EarlyStopping

n_features = X_LSTM.shape[-1]  # 特征数量
n_units = 100     # LSTM层单元数量

model = keras.Sequential()
model.add(layers.LSTM(n_units, input_shape=(1, n_features)))
model.add(layers.Dense(1))  # 因为您的目标输出是单个值

model.compile(optimizer=tf.keras.optimizers.Adam(1e-3), loss='mse', metrics=['mse']) # 选择合适的优化器和损失函数
model.summary()
early_stopping = EarlyStopping(monitor='val_loss', patience=10, min_delta=0.0001, mode='min', verbose=1, restore_best_weights=True)

history = model.fit(x_train, y_train, epochs=100, batch_size=32, validation_split=0.2, callbacks=[early_stopping])

model.save('model/LSTM')

fig = plt.figure(figsize=(5,5), dpi=100)
plt.plot(history.epoch, history.history['mse'], label = 'Training', color='r')
plt.plot(history.epoch, history.history['val_mse'], label = 'Validation', color='b')
plt.title('Training history')
plt.xlabel('Epoch')
plt.ylabel('Mean Square Error')
plt.legend(loc='upper right')
plt.grid()
plt.xlim([-5,70])
plt.savefig('fig4.png')
# plt.show()

model.evaluate(x_train, y_train);model.evaluate(x_test, y_test)
\end{lstlisting}

\subsection{predictions\_LSTM.py}
\begin{lstlisting}
import tensorflow as tf
import tensorflow.keras as keras

import pandas as pd
import numpy as np
import sys, math
import matplotlib.pyplot as plt

import preprocessing

plt.rc('font', size=18); plt.rcParams['figure.constrained_layout.use'] = True

def feature_all_time_series(q = 3, lag = 3):
    def feature_time_series(q, lag, plot, y, t, id, dt):
    # q-step ahead prediction
        stride = 1

        # m = math.floor(30*7*24*60*60 / dt) # number of samples per month
        w = math.floor(7*24*60*60 / dt) # number of samples per week
        d = math.floor(24*60*60 / dt)

        len = y.size - w - lag * w - q

        XX = y[q: q+len: stride]
            
        for i in range(1, lag):
            temp = y[i*w+q: i*w+q+len: stride]
            XX = np.column_stack((XX, temp))

        for i in range(0, lag):
            temp = y[i*d+q: i*d+q+len: stride]
            XX = np.column_stack((XX, temp))

        for i in range(0, lag):
            temp = y[i+q: i+q+len: stride]
            XX = np.column_stack((XX, temp))

        yy = y[lag*w+w+q: lag*w+w+q+len: stride]
        tt = t[lag*w+w+q: lag*w+w+q+len: stride]
        iidd = id[lag*w+w+q: lag*w+w+q+len: stride]
        return XX, yy, tt, iidd

    for idx, idd in enumerate(station_info):
        idd = int(idd)
        if idd == 1:
            X, Y, T, ID = feature_time_series(q, lag, True, y[id==idd], t[id==idd], id[id==idd], dt)
        else:
            X0, y0, t0, id0 = feature_time_series(q, lag, True, y[id==idd], t[id==idd], id[id==idd], dt)
            X = np.row_stack((X, X0))
            Y = np.concatenate((Y, y0))
            T = np.concatenate((T, t0))
            ID = np.concatenate((ID, id0))
    X.shape, Y.shape, T.shape, ID.shape
    return X, Y, T, ID

from sklearn.utils import resample

def bootstrap_evaluate(model, data, labels, n_iterations=100, sample_size=0.2):
    scores = list()
    n_size = int(len(data) * sample_size)
    
    for i in range(n_iterations):
        # 准备Bootstrap样本
        indices = np.random.randint(0, len(data), n_size)
        sample_data, sample_labels = data[indices], labels[indices]
        
        # 评估模型
        loss, accuracy = model.evaluate(sample_data, sample_labels, verbose=0)
        scores.append(accuracy)
    
    # 分析Bootstrap结果
    mean_score = np.mean(scores)
    std_dev = np.std(scores)
    return scores


def main():
    model = tf.keras.models.load_model('model/LSTM')
    
    
    isolate_station = False
    y, t, id, dt, station_info = preprocessing.main('pre', isolate_station)
    y_pre = y; t_pre = t; id_pre = id; dt_pre = dt
    X_pre, Y_pre, T_pre, ID_pre = feature_all_time_series(q = 3, lag = 3)
    # 在训练之前,确保 x_train 和 x_test 的形状是 (样本数, 1, 特征数)
    X_LSTM = X_pre.reshape((X_pre.shape[0], 1, 9))
    X_LSTM.shape
    
    
    
    isolate_station = False
    y, t, id, dt, station_info = preprocessing.main('dur', isolate_station)
    y_dur = y; t_dur = t; id_dur = id; dt_dur = dt
    X_dur, Y_dur, T_dur, ID_dur = feature_all_time_series(q = 3, lag = 3)
    # 在训练之前,确保 x_train 和 x_test 的形状是 (样本数, 1, 特征数)
    X_LSTM_dur = X_dur.reshape((X_dur.shape[0], 1, 9))
    X_LSTM_dur.shape
    
    
    
    isolate_station = False
    y, t, id, dt, station_info = preprocessing.main('post', isolate_station)
    y_post = y; t_post = t; id_post = id; dt_post = dt
    X_post, Y_post, T_post, ID_post = feature_all_time_series(q = 3, lag = 3)
    # 在训练之前,确保 x_train 和 x_test 的形状是 (样本数, 1, 特征数)
    X_LSTM_post = X_post.reshape((X_post.shape[0], 1, 9))
    X_LSTM_post.shape
    
    
    model.evaluate(X_LSTM, Y_pre)
    scores_pre = bootstrap_evaluate(model, X_LSTM, Y_pre)
    
    model.evaluate(X_LSTM_dur, Y_dur)
    scores_dur = bootstrap_evaluate(model, X_LSTM_dur, Y_dur)
    
    model.evaluate(X_LSTM_post, Y_post)
    scores_post = bootstrap_evaluate(model, X_LSTM_post, Y_post)
    
    print("Mean and std of pre-pandemic are:")
    print(np.mean(scores_pre),np.std(scores_pre))
    
    print("Mean and std of pandemic are:")
    print(np.mean(scores_dur), np.std(scores_dur))
    
    print("Mean and std of post-pandemic are:")
    print(np.mean(scores_post), np.std(scores_post))
    
    pred_dur = model.predict(X_LSTM_dur)
    pred_post = model.predict(X_LSTM_post)
    
    
    figure = plt.figure(figsize=(7,5), dpi=300)
    plt.scatter(T_dur, Y_dur, s=1, c='b', alpha=0.8, label='Samples')
    plt.scatter(T_dur, pred_dur , s=1,c='tomato',alpha=0.8, label='Predictions')
    plt.legend()
    plt.grid()
    plt.ylim([-11,11])
    plt.xlabel("Days (2020-03-13 to 2022-01-27)")
    plt.ylabel("Normailzed Values")
    plt.title("Predictions & Samples of Pandemic Period")
    plt.savefig('fig5.png')
    plt.show()
    
    figure = plt.figure(figsize=(7,5), dpi=300)
    plt.scatter(T_post, Y_post, s=1, c='b', alpha=0.8, label='Samples')
    plt.scatter(T_post, pred_post , s=1,c='tomato',alpha=0.8, label='Predictions')
    plt.legend(loc='lower right')
    plt.grid()
    plt.ylim([-10,10])
    plt.xlabel("Days (2022-01-27 to 2023-12-25)")
    plt.ylabel("Normailzed Values")
    plt.title("Predictions & Samples of Pandemic Period")
    plt.savefig('fig6.png')
    plt.show()
\end{lstlisting}