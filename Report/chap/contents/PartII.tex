
    
\chapter{Q\normalsize{UESTIONS}}
\titleformat{\paragraph}[runin]
{\normalsize\itshape\bfseries}{\theparagraph}{1em}{}

\thequestion What is a ROC curve? How can it be used to evaluate the performance of a classifier
compared with a baseline classifier? Why would you use an ROC curve instead of a
classification accuracy metric? 
\refstepcounter{question}

\paragraph{Answer}
For the two-feature classification task
which the data are labeled only two type of classes. 
Conventional choice is linear model $y = \theta^{\text{T}} X$ 
which is to predict whether label associated with feature vector $X$ is $1$ or $−1$ 
generally, with decision boundary $\alpha$.
Commonly used, is the true-positive (TP), false-positive (FP) rates which mean 
the model predicted the positive label right and false.
As vary the parameter $\alpha$ the balance of true-positive and false-positive rates.
A ROC curve is a plot of true positive rate vs false
positive rate.

\paragraph{Answer}
The accuracy of predictions is $\frac{TP}{FP+TP}$, thus the idea classifier 
has 100\% TP and 0\% FP. 
The idea classifier gives a point $(0,1)$ at the upper-left conner of 
ROC plot. 
The baseline classifier which just predict the most frequent feature or 
just random feature, gives a point on the $y=x$ line.
So the way of ROC evaluate the performance of classifier is to 
determine how close of the ROC curve of the classifier to the upper-left conner of 
the region $[0,1]\times [0,1]$.

\paragraph{Answer}
Using accuracy as metrics have problem when handle imbalanced datasets.
If there are 90\% samples are labels as A 10\% as B, the most frequency 
baseline model will have 90\% accuracy.
However, this doesn't necessarily indicate good performance in recognizing the minority class
which means accuracy is not a good assessment.
ROC curves provide a more comprehensive performance assessment 
by considering both the true positive rate and false positive rate.

The ROC curve provide a full view of performance under various thresholds $\alpha$.
While accuracy is based on a specific decision threshold (just one point), 
which may not always be apparent, 
especially in cases where some classes are more critical than others. 
The ROC curve shows how various thresholds affect the performance of classifier.

Also the ROC curve provide full information to designers which allows to 
choice the threshold $\alpha$ with different tolerance of FT and TP.
Besides that, the ROC curve make it is easier to compare the performance of 
different classifiers by visually approximately determine which classifier 
with which tolerance $\alpha$ is they needed.

\vspace{1em}
\thequestion Give two examples of situations where a linear regression would give inaccurate
predictions. Explain your reasoning and what possible solutions you would adopt in each
situation.
\refstepcounter{question}

\paragraph*{Answer}
There are many cases that linear regression models are more likely to give inaccurate
predictions.
\paragraph{Example}
Linear model assume that the liner relationships among the samples which
are independent and dependent related. 
Such as the samples sets $\{(x_{k}, y_k)\}_{k=1}^{N}$ where 
$X = \{x_k\}_{k=1}^{N}$ and $y_{k} = f(x_k) = x_k^2$ (quadratic related).
The linear model $\widehat{f}(x) = \theta_1 x + \theta_0$ is not able to capture the non-linear relationship in this 
sample sets.
Therefore, it will make inaccurate predictions, although it was trained.
\paragraph{Solution}
In this case, if keeping using linear model, the solution commonly is to 
feature engineering the sample set $\{(x_{k}, y_k)\}_{k=1}^{N}$ with $2$ polynomial features.
The new sample set will be composed by $(x_{k}, x^2_{k}, y_k)$. 
Furthermore, the linear model $\widehat{f}$ is a equation below
\[
    \widehat{y_k} = \widehat{f}(x_{k}, x^2_{k}) 
    = \theta_2 x_k^2 + \theta_1 x_k + \theta_0
\]
At this case, the new linear model trained on featured sample set could learn the quadratic
relationships among the sample set.

\paragraph{Example}
Linear model, also other models, have higher chance to make inaccurate predictions
as long as the samples are not preprocessed well. 
Such as the presence of out-liner, missing some information and unbalanced samples.
Take the presence of out-liner as an example.

Since linear regression models are sensitive to the samples which are 
significantly different with others, 
especially in the independent variables (predictors).
They can have a disproportionately large influence on the line of best fit. 
This can skew the results of the linear model, leading to inaccurate predictions.

\vspace{1em}
\thequestion The term 'kernel' has different meanings in SVM and CNN models. Explain the two
different meanings. Discuss why and when the use of SVM kernels and CNN kernels is
useful, as well as mentioning different types of kernels.
\refstepcounter{question}
\paragraph{Explainations}
In Kernel SVMs, 
a kernel is a \underline{function} $\kappa$
used for transforming the data into a higher dimension 
where a linear separator might be found. 
It can be represented as $\kappa(x_{i}, x_{j}) = \phi(x_i)^{\text{T}}\phi(x_j)$
which is the dot product of data points 
in a transformed feature space 
enabling the SVM to classify data that 
is not linearly separable in the original space.
The primary purpose of using a kernel in SVMs 
is to solve nonlinear relationships 
by applying a linear classification approach. 
Common kernels include linear combination, polynomial, and sigmoid ($\sigma$). 

In CNNs, a kernel (or filter) refers to a small \underline{matrix} used to extract 
features from input data (especially for image classification tasks such as 
ImageNet Challenge) through a process called convolution and collectively as 
convolution layer. 
The kernel slides (shift) over the input data, 
performing element-wise multiplication 
and sum subsequently.
It effectively capturing patterns such as 
edges, shapes, and textures. 
Each kernel in a CNN is trained to identify a specific type 
of feature in the input, and through multiple layers of 
convolutions and other operations, 
CNNs can recognize complex patterns in larger structures such 
as object detection or facial recognition datasets.

\paragraph{Reasons}
In many real-world scenarios, data is not linearly separable. 
Kernel SVMs allow these datasets to be transformed 
into a higher-dimensional space without loosing no-linear relationships,
where a linear separator can be found. 
Moreover, the kernel can be applied in any linear model not just for SVMs.
The model $\widehat{f}$ in previous question is an example.

In real-word tasks, the input sample is commonly too large to evaluate 
in single iteration, such as a 4K (4360*2160 pixels) image captioning tasks.
However, by using a series of various size of kernels 
(for RGB image data, commonly 3*3*3, 5*5*3, 7*7*3 small kernels, 
32*32*3, 64*64*3, 96*96*3 large kernels), training them in the process of 
sliding through data, the computational demands are reduced significantly.
Besides that, with different size of kernels, the kernels can learn 
various patterns inside given data which is more complicated to achieve using 
single matrix multiplication. 
Moreover, kernels with different size can learning different features, such 
as using 32*32*3 kernel to learn the feature both in RGB channels, and 
32*32*1 kernel to lean the feature in one channel. 
By combining various type of kernels, the CNN model can learn 
various features in data effectively.

\vspace{1em}
\thequestion In k-fold cross-validation, a dataset is resampled multiple times. What is the idea behind
this resampling i.e. why does resampling allow us to evaluate the generalization
performance of a machine learning model. Give a small example to illustrate. Discuss
when it is and it is not appropriate to use k-fold cross-validation.

\paragraph{Answer}
In real-world case, the researchers want to maximize the utilization of 
collected data. 
However, splitting the dataset into training dataset and testing dataset,
a large portion of data will not used for training which 
have higher chance to be particularly problematic with small datasets. 
Also called the model has not sufficient generalization performance.
On the contrast, $k$-fold cross validation is an approach that 
ensure all sample are utilized for training, but by separating the dataset into 
$k$ parts. 
Since using more data for training, the model will have better generalization
performance.

Besides that, this approach helps researchers identify the 
data-driven issues like over-fitting and under-fitting, 
since the model trained on more data.
After the model trained on better processed dataset, the model will 
have better performance.

\paragraph{Answer}
$k$-fold cross-validation 
is appropriate for small to medium datasets, model selection, 
hyperparameter tuning, and when a robust estimate of model performance is required. 

$k$-fold cross-validation is not suitable for very large datasets due to 
computation resource demands, extremely imbalanced datasets, 
or data with grouped observations, 
as it may disrupt the inherent structure of the data.
